\documentclass[UTF8,12pt]{article} % 12pt 为字号大小
\usepackage{amssymb,amsfonts,amsthm}
%\usepackage{fontspec,xltxtra,xunicode}
%\usepackage{times}
\usepackage{amsmath,bm,braket}
\allowdisplaybreaks[4]
\usepackage{mdwlist}
\usepackage[colorlinks,linkcolor=blue]{hyperref}
\usepackage{cleveref}
\usepackage{float}
\usepackage{enumerate}
\usepackage{extarrows}
\usepackage[ruled,linesnumbered]{algorithm2e}
%\numberwithin{equation}{section}

%----------
% 定义中文环境
%----------

\usepackage{xeCJK}
\setCJKmainfont[BoldFont={Heiti SC Light},ItalicFont={Kaiti SC Regular}]{Songti SC Regular}
\setCJKsansfont{Heiti SC Light}
\setCJKfamilyfont{song}{Songti SC Regular}
\setCJKfamilyfont{zhhei}{Heiti SC Light}
\setCJKfamilyfont{zhkai}{Kaiti SC Regular}
\setCJKfamilyfont{zhfs}{STFangsong}
\setCJKfamilyfont{zhli}{Libian SC Regular}
\setCJKfamilyfont{zhyou}{Yuanti SC Regular}

\newcommand*{\songti}{\CJKfamily{zhsong}} % 宋体
\newcommand*{\heiti}{\CJKfamily{zhhei}}   % 黑体
\newcommand*{\kaiti}{\CJKfamily{zhkai}}  % 楷体
\newcommand*{\fangsong}{\CJKfamily{zhfs}} % 仿宋
\newcommand*{\lishu}{\CJKfamily{zhli}}    % 隶书
\newcommand*{\yuanti}{\CJKfamily{zhyou}} % 圆体

%----------
% 版面设置
%----------
%首段缩进
\usepackage{indentfirst}
\setlength{\parindent}{2em}

%行距
\renewcommand{\baselinestretch}{1.2} % 1.2倍行距

%页边距
\usepackage[a4paper]{geometry}
\geometry{verbose,
  tmargin=2cm,% 上边距
  bmargin=2cm,% 下边距
  lmargin=2.5cm,% 左边距
  rmargin=2.5cm % 右边距
}


%----------
% 其他宏包
%----------
%图形相关
\usepackage[x11names]{xcolor} % must before tikz, x11names defines RoyalBlue3
\usepackage{graphicx}
\graphicspath{{figures/}}
\usepackage{pstricks,pst-plot,pst-eps}
\usepackage{subfig}
\def\pgfsysdriver{pgfsys-dvipdfmx.def} % put before tikz
\usepackage{tikz}

%原文照排
\usepackage{verbatim}

%网址
\usepackage{url}

%----------
% 定理、习题与解答环境
%----------
%定理环境
\usepackage[most]{tcolorbox}
\newtcbtheorem[number within=section]{theorem}{Theorem}{
     enhanced,
     breakable,
     sharp corners,
     attach boxed title to top left={
       yshifttext=-1mm
     },
     colback=blue!4!white,
     colframe=blue!75!black,
     fonttitle=\bfseries,
     boxed title style={
       sharp corners,
       size=small,
       colback=blue!75!black,
       colframe=blue!75!black,
     } 
}{theorem}

\newtcbtheorem[number within=section]{definition}{Definition}{
     enhanced,
     breakable,
     sharp corners,
     attach boxed title to top left={
       yshifttext=-1mm
     },
     colback=blue!4!white,
     colframe=blue!75!black,
     fonttitle=\bfseries,
     boxed title style={
       sharp corners,
       size=small,
       colback=blue!75!black,
       colframe=blue!75!black,
     } 
}{definition}

\newtcbtheorem[number within=section]{corollary}{Corollary}{
     enhanced,
     breakable,
     sharp corners,
     attach boxed title to top left={
       yshifttext=-1mm
     },
     colback=blue!4!white,
     colframe=blue!75!black,
     fonttitle=\bfseries,
     boxed title style={
       sharp corners,
       size=small,
       colback=blue!75!black,
       colframe=blue!75!black,
     } 
}{corollary}

\newtcbtheorem[number within=section]{myboxes}{Box}{
     enhanced,
     breakable,
     sharp corners,
     attach boxed title to top left={
       yshifttext=-1mm
     },
     %colback=white,
     colframe=black!75!white,
     fonttitle=\bfseries,
     boxed title style={
       sharp corners,
       size=small,
       colback=black!75!white,
       colframe=black!75!white,
     } 
}{myboxes}

%习题环境
\newtcbtheorem[number within=section]{exercise}{Problem}{
     enhanced,
     breakable,
     sharp corners,
     attach boxed title to top left={
       yshifttext=-1mm
     },
     colback=white,
     colframe=black,
     fonttitle=\bfseries,
     boxed title style={
       sharp corners,
       size=small,
       colback=black,
       colframe=black,
     } 
}{Problem}

%解答环境
\ifx\proof\undefined\
\newenvironment{proof}[1][\protect\proofname]{\par
\normalfont\topsep6\p@\@plus6\p@\relax
\trivlist
\itemindent\parindent
\item[\hskip\labelsep
\scshape
#1]\ignorespaces
}{%
\endtrivlist\@endpefalse
}
\fi

\renewcommand{\proofname}{\it{Solution}}

%==========
% 正文部分
%==========

\begin{document}

\title{Chapter 5: Approximation methods}
\author{Yuquan Chen, Liu Zhu}
\date{2019/05/21} % 若不需要自动插入日期,则去掉前面的注释;{ } 中也可以自定义日期格式
\maketitle

\section{Time-independent perturbation theory}

We start with the problem really close to a solved problem, then we can use the solution at hand to do approximation. 

\begin{myboxes}{Recap for Taylor expansion}{}
\begin{equation}
   f(x)\simeq f(x_0)+f'(x_0)(x-x_0)+\cdots+\frac{f^n(x_0)}{n!}(x-x_0)^n+\cdots . 
\end{equation}
If keep all the way to $(x-x_0)^2$, then we can do a fit with a polynomial function.
\end{myboxes}

For quantum mechanics:
\begin{equation}
    H=H_{0} + H'
\end{equation}
$H_0$ has a known solution for eigen energy $E^{(0)}_{n}$ , and eigenstates $\{\ket{n^{(0)}}\}$ :
\begin{equation}
    H_0\ket{n^{(0)}}=E^{(0)}_n\ket{n^{(0)}}
\end{equation}
$H'$ is perturbational part of Hamiltonian, it can express as:
\begin{equation}
    H'=\lambda V
\end{equation}
where $\lambda$ is a number, and $\lambda\ll1$ . $V$ is another part of Hamiltonian.

Then we try to find $H\ket{n}=E_n\ket{n}$ . $E_n$ must be a function of $H$ ,which is a function of $\lambda$ , so as $\ket{n}$ . Therefore we can make a Taylor expansion of $E_n(\lambda)$ and $\ket{n(\lambda)}$ :
\begin{equation}
    \begin{aligned}
        E_n(\lambda)&=E_n^{(0)}+\lambda E^{(1)}_n+\lambda^2 E^{(2)}_n+\cdots , 
        \\ 
        \ket{N(\lambda)}&=\ket{n^{(0)}}+\lambda \ket{n^{(1)}}+\lambda^2 \ket{n^{(2)}}+\cdots , 
    \end{aligned}
\end{equation}
where $\lambda E^{(1)}_n$ is first order energy shift. $\lambda E^{(2)}_n$ is second order energy shift. Plug the expansion back to $H\ket{n}=E\ket{n}$ with $H=H_0+\lambda V$:
\begin{multline}
    (H_0+\lambda V)\cdot (\ket{n^{(0)}}+\lambda \ket{n^{(1)}}+\lambda^2\ket{n{(2)}}+\cdots)\\=(E^{(0)}_n+\lambda E^{(1)}+\lambda^2E^{(2)}+\cdots)\cdot (\ket{n^{(0)}}+\lambda \ket{n^{(1)}}+\lambda^2\ket{n^{(2)}}+\cdots) , 
\end{multline}
do not contain $\lambda$, Equation (1.6) $\Rightarrow H_0\ket{n^{(0)}}=E^{(0)}_n\ket{n^{(o)}}$, for $\lambda^1$ term we can get:
\begin{equation}
    H_0\ket{n^{(1)}}+V\ket{n^{(0)}}=E^{(0)}_n\ket{n^{(1)}}+E^{(1)}_n\ket{n^{(0)}} , 
\end{equation}
apply $\bra{n^{(0)}}$ to the left:
\begin{equation}
    \boxed{ \braket{n^{(0)}|V|n^{(0)}}=E^{(1)}_n , }  
\end{equation}
which is {\bf first order pertubrbation for $E_n$}, plug in to $E_n=E^{(0)}_n+\lambda E^{(1)}_n $, we can get:
\begin{equation}
    \begin{aligned}
        E_n= E^{(0)}_n+ \braket{n^{(0)}|\lambda V|n^{(0)}} \to \boxed {E_n=E^{(0)}+\braket{n^{(0)}|H'|n^{(0)}} . }
    \end{aligned}    
\end{equation}

For $H=H_0+\lambda V$ , we can express it in the energy representation in $\set{\ket{n^{(0)}}}$:
\[ H_0 = 
\begin{pmatrix}
     E^{(0)}_0& 0         & 0         & \ldots    & 0         \\
     0        & E^{(0)}_1 & 0         & \ldots    & 0         \\
     0        & 0         & E^{(0)}_2 & \ldots    & 0         \\
     \vdots   & \vdots    & \vdots    & \ddots    & \vdots    \\
     0        & 0         & 0         & \ldots    & E^{(0)}_n \\
\end{pmatrix} . \]
$\lambda V$ in the same basis can express as:
\[ \lambda V = 
\begin{pmatrix}
    \lambda V_{11} & \lambda V_{12} & \ldots & \lambda V_{1n} \\
    \lambda V_{21} & \lambda V_{22} & \ldots & \lambda V_{2n} \\
    \vdots         & \vdots         & \ddots & \vdots         \\
    \lambda V_{n1} & \lambda V_{n2} & \ldots & \lambda V_{nn} \\
\end{pmatrix} . \]
For $n=0$, $E_0\simeq E^{(0)}_0+\lambda V_{11}$. First order perturbation is just n row, n colunm for matrix $H_0+\lambda V$ in $\set{\ket{n^{(0)}}}$ basis.

Now we try to find first order pertubrbation for state. From Equation (1.7) and (1.8):
\begin{equation*}
    \begin{aligned}
        H_0\ket{n^{(1)}}+V\ket{n^{(0)}}&=E^{(0)}_n\ket{n^{(1)}}+E^{(1)}_n\ket{n^{(0)}} , \\
        \braket{n^{(0)}|V|n^{(0)}}&=E^{(1)}_n ,
    \end{aligned}
\end{equation*}
we try to solve $\ket{n^{(1)}}$, we can apply $\bra{k^{(0)}}$ to left, with $k\neq n$, then $\braket{k^{(0)}|n^{(0)}}=0$. we also require $E^{(0)}\neq E^{(0)}$. We can get 
\begin{equation} 
    \begin{aligned}
    \braket{k^{(0)}|H_0|n^{(1)}}+\braket{k^{(0)}|V|n^{(0)}}&=E^{(0)}_n\braket{k^{(0)}|n^{(1)}}+E^{(1)}_n\braket{k^{(0)}|n^{(0)}} , \\
    E^{(0)}_k\braket{ k^{(0)} | n^{(1)} }+\braket{k^{(0)}|V|n^{(0)}}&=E^{(0)}_n\braket{k^{(0)} | n^{(1)}} . 
    \end{aligned}
\end{equation}
The term $\braket{k^{(0)} | n^{(1)}}$ is a inner product between an unknown state $\ket{n^{(1)}}$ and a state in the known basis, it called ``amplitude''.

We want to solve the state $\ket{n^{(1)}}$ , from the {\bf Superposition Principle}, we know that $\ket{n^{(1)}}=\sum_k{C_k \ket{k^{(0)}}}$ , since $\set{\ket{k^{(0)}}}$ form a basis. If we can solve every $C_k$ , the state $\ket{n^{(1)}}$ is known. We can easily observe that $\braket{k^{(0)} | n^{(1)}} \to \sum_k{ C_k \braket{ k^{(0)} | n^{(0)} } } \to \sum_k{ C_k \delta_{k,n} } \to C_k$ , so all we need is to solve $\braket{ k^{(0)} | n^{(1)} }$ . 
\newpage From Equation (5.10) we can solve $\braket{ k^{(0)} | n^{(1)} }$ , which qual to $C_k$ :
\begin{equation}
    \boxed{ C_k = \frac{\braket{k^{(0)}|V|n^{(0)}}}{E^{(0)}_n-E^{(0)}_k} , }
\end{equation}
notice that $k\neq n$ . Therefore, we can get the {\bf first order perturbation for state $\ket{n}$ }
\begin{equation}
    \boxed{ \ket{n^{(1)}} = \sum_{k\neq m} \frac{ \braket{ k^{(0)}|V | n^{(0)} } }{ E^{(0)}_n - E^{(0)}_k } \ket{ k^{(0)}}, }
\end{equation}
plug it and $H'=\lambda V$ back to $ \ket{n} = \ket{n^{(0)}} + \lambda \ket{n^{(1)}} $ we get 
\begin{equation}
    \boxed{ \ket{n} = \ket{n^{(0)}} + \sum_{k\neq m} \frac{ \braket{ k^{(0)}|H'| n^{(0)} } }{ E^{(0)}_n - E^{(0)}_k } \ket{ k^{(0)}} , }
\end{equation}
where $\braket{k^{(0)}|H'|n^{(0)}}$ can be the matrix element in the $k$ row and $n$ colum of matrix $H'$ in the basis of $\set{\ket{n^{(0)}}}$ :
\[ H_0+H' \xrightarrow{matirx \hspace{1ex} in \hspace{1ex} \set{\ket{n^{(0)}}}} 
\begin{pmatrix}
    &  &  &  \\
    & E^{(0)}_{ka} & H^{(0)}_{kn} & \\
    &  & E^{(0)}_{bn} & \\
    &  & & \\
\end{pmatrix} . \] 
\subsection*{Example: Spin - $\dfrac{1}{2}$ system}
We have
\begin{equation*}
    H_0=\Omega \sigma_z \to \begin{pmatrix}
     \Omega & 0 \\
     0 & \Omega \\   
    \end{pmatrix} ; \hspace{2em}
    H'=\lambda \sigma_x \to \begin{pmatrix}
        0 & \lambda \\
        \lambda & 0 \\   
       \end{pmatrix} ,
\end{equation*}
The Hamiltonian is
\begin{equation*}
    H=H_0+H' \to \begin{pmatrix}
        \Omega & \lambda \\
        \lambda & \Omega \\
    \end{pmatrix} , and \hspace{1em} \Omega \gg \lambda
\end{equation*}
we can solve eigenenergy 
\begin{equation*}
    E_1=\Omega + \frac{\lambda^2}{2\Omega} , \hspace{1em} E_2=-\Omega - \frac{\lambda^2}{2\Omega} ,
\end{equation*}
and the eigenstates
\begin{equation*}
    \ket{\psi}_1= \begin{pmatrix}
        1-\frac{\lambda^2}{2\Omega^2} \\
        \frac{\lambda}{2\Omega} \\ 
    \end{pmatrix} ,
\end{equation*}
\begin{equation*}
    \ket{\psi}_2= \begin{pmatrix}
        -\frac{\lambda}{2\Omega} \\ 
        1-\frac{\lambda^2}{2\Omega^2} \\
    \end{pmatrix} .
\end{equation*}

\end{document}